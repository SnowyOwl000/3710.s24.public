\documentclass{article}

% page uses 1-inch margins on all sides
\usepackage[margin=1in]{geometry}

% these are different fonts
\usepackage{arev,tgadventor,tgcursor}
\usepackage[semibold]{sourcecodepro}
% use TeX Gyre Adventor as the default font
\renewcommand{\rmdefault}{qag}
% sans serif is the base font
\renewcommand*\familydefault{\sfdefault}

\usepackage{amssymb,amsmath}
\usepackage{amsthm}

\begin{document}
% \noindent = don't indent the start of the next paragraph
% \Huge = use a very large font for this text
\noindent{\Huge Logic Symbols}\\
% make a box as wide as an entire line of text
% inside that box, make a rule (solid line) as wide as the box, and 0.4 points
% thick (a thin line stretching across the page)
\noindent\makebox[\linewidth]{\rule{\linewidth}{0.4pt}}\\

First, note that the fonts are different. I tend to use this particular font
--- \TeX{} Gyre Adventor --- for text. Check out the preamble in
{\tt logicsymbols.tex} to see how I change the font.\\

This document presents the basic propositional and predicate logic symbols we
use, as well as an environment you can use to format ``two-column'' proofs. To
begin, here is a truth table showing one of DeMorgan's laws:
\begin{center}
  % note the double | to add some separation between columns
  \begin{tabular}{|c|c||c|c|c|c|c||c|}
    \hline
    % \lnot = logical not symbol
    % \lor = logical or symbol (aka \vee)
    % \land = logical and symbol (aka \wedge)
    % \iff = if-and-only-if (biconditional)
    $p$ & $q$ & $\lnot p$ & $\lnot q$ & $p\lor q$ & $\lnot(p\lor q)$ & $\lnot
    p\land\lnot q$ & $\lnot(p\lor q)\iff\lnot p\land\lnot q$\\ \hline\hline
    T & T & F & F & T & F & F & T\\ \hline
    T & F & F & T & T & F & F & T\\ \hline
    F & T & T & F & T & F & F & T\\ \hline
    F & F & T & T & F & T & T & T\\ \hline
  \end{tabular}
\end{center}

This shows the basic logic operations and, or and not, as well as the
biconditional ``if and only if.'' Check out the source code to see the commands
for the symbols. Note they are all in math mode.\\

Here's one more table showing an if-then proposition.
\begin{center}
  % note the double | to add some separation between columns
  \begin{tabular}{|c|c||c|c|c||c|}
    \hline
    % \lnot = logical not symbol
    % \lor = logical or symbol (aka \vee)
    % \land = logical and symbol (aka \wedge)
    % \iff = if-and-only-if (biconditional)
    $p$ & $q$ & $\lnot p$ & $(\lnot p)\lor q$ & $p\to q$ & $(\lnot p)\lor q\iff
    p\to q$\\ \hline\hline
    T & T & F & T & T & T\\ \hline
    T & F & F & F & F & T\\ \hline
    F & T & T & T & T & T\\ \hline
    F & F & T & T & T & T\\ \hline
  \end{tabular}
\end{center}

Next are the universal and existential quantifiers, $\forall$ and $\exists$.
They are always followed by a variable and eventually a propositional function,
such as in $\forall x\,P(x)$, $\exists y\,Q(y)$ and $\forall x\exists y\,
R(x,y)$.\\
% note the "thin space" \, between the quantified variable and the function
% also note: \/ adjusts spacing when switching from italicized text to upright
% text, so that the two character at the junction don't overlap at the top.
\emph{Note}\/: It's common to add a small (thin) space between the quantifier
variable and the function; see the source to see how that's done.\\

Next, here's an example of the {\tt \textbackslash align} environment, which
combines math mode with a two-column layout, which is very handy for proofs. For
this example, we'll revisit the Socrates example from the book:
\begin{center}
  All men are mortal. Socrates is a man. Therefore, Socrates is mortal.
\end{center}

To turn this into logical statements, let $H(x)$ be ``$x$ is a (hu)man'' and let
$M(x)$ represent ``$x$ is mortal.'' Then, we have two premises:
\begin{enumerate}
  \item $\forall x\,(H(x)\to M(x))$ --- for all entities $x$, if $x$ is human,
  then $x$ is mortal.
  \item $H({\rm Socrates})$ --- Socrates is a human.
\end{enumerate}

Here is the proof that Socrates is mortal. Again, check out the source code.
\begin{align}
  % \qquad = two "quads" of space
  \forall x\,(H(x)\to M(x)) & \qquad\textrm{Premise 1}\\
  H({\rm Socrates})\to M({\rm Socrates}) & \qquad\textrm{(1), Universal
  Instantiation}\\
  H({\rm Socrates}) & \qquad\textrm{Premise 2}\\ \hline
  % \therefore is the triangular-three-dot symbol
  \therefore M({\rm Socrates}) & \qquad\textrm{(2) and (3), \emph{modus ponens}}
\end{align}
\newpage
Some notes about this example:
\begin{itemize}
  \item In the {\tt \textbackslash align} environment, every line is numbered.
  You can refer to these numbers elsewhere in the document. To do so, add labels
  and references; see the example in the math mode documents.\\
  \emph{Note}\/: I manually referred to lines $1$, $2$ and $3$. That always
  works, but can be tedious with a large document.
  \item Use an ampersand ({\tt \&}) to switch from column 1 to column 2.
  \item Use two backslashes to end a line. It is not necessary to end the last
  line.
  \item I used {\tt \textbackslash qquad} to add space between the columns. Try
  removing it and render the document to see what it's like without the space.
  \item {\tt \textbackslash hline} draws a horizontal line across the page above
  the current line, which is useful at the end of the proof.
  \item There is also an {\tt \textbackslash align*} environment which is
  identical to {\tt \textbackslash align}, except the lines are not numbered.
  Sometimes that is preferable (but probably not for our purposes here).
\end{itemize}

Finally, there is the one more logic operator, the \emph{exclusive or}, often
abbreviated as \emph{xor}. Its symbol is a plus sign inside a circle. Here is a
short example, demonstrating that exclusive or is the opposite of if and only
if:
\vskip12pt

\begin{tabular}{|c|c||c|c|}
  \hline
  $p$ & $q$ & $p\oplus q$ & $p\iff q$\\ \hline\hline
  T & T & F & T\\ \hline
  T & F & T & F\\ \hline
  F & T & T & F\\ \hline
  F & F & F & T\\ \hline
\end{tabular}

\end{document}
